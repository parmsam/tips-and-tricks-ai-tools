% Options for packages loaded elsewhere
\PassOptionsToPackage{unicode}{hyperref}
\PassOptionsToPackage{hyphens}{url}
\PassOptionsToPackage{dvipsnames,svgnames,x11names}{xcolor}
%
\documentclass[
  letterpaper,
  DIV=11,
  numbers=noendperiod]{scrreprt}

\usepackage{amsmath,amssymb}
\usepackage{lmodern}
\usepackage{iftex}
\ifPDFTeX
  \usepackage[T1]{fontenc}
  \usepackage[utf8]{inputenc}
  \usepackage{textcomp} % provide euro and other symbols
\else % if luatex or xetex
  \usepackage{unicode-math}
  \defaultfontfeatures{Scale=MatchLowercase}
  \defaultfontfeatures[\rmfamily]{Ligatures=TeX,Scale=1}
\fi
% Use upquote if available, for straight quotes in verbatim environments
\IfFileExists{upquote.sty}{\usepackage{upquote}}{}
\IfFileExists{microtype.sty}{% use microtype if available
  \usepackage[]{microtype}
  \UseMicrotypeSet[protrusion]{basicmath} % disable protrusion for tt fonts
}{}
\makeatletter
\@ifundefined{KOMAClassName}{% if non-KOMA class
  \IfFileExists{parskip.sty}{%
    \usepackage{parskip}
  }{% else
    \setlength{\parindent}{0pt}
    \setlength{\parskip}{6pt plus 2pt minus 1pt}}
}{% if KOMA class
  \KOMAoptions{parskip=half}}
\makeatother
\usepackage{xcolor}
\setlength{\emergencystretch}{3em} % prevent overfull lines
\setcounter{secnumdepth}{5}
% Make \paragraph and \subparagraph free-standing
\ifx\paragraph\undefined\else
  \let\oldparagraph\paragraph
  \renewcommand{\paragraph}[1]{\oldparagraph{#1}\mbox{}}
\fi
\ifx\subparagraph\undefined\else
  \let\oldsubparagraph\subparagraph
  \renewcommand{\subparagraph}[1]{\oldsubparagraph{#1}\mbox{}}
\fi

\usepackage{color}
\usepackage{fancyvrb}
\newcommand{\VerbBar}{|}
\newcommand{\VERB}{\Verb[commandchars=\\\{\}]}
\DefineVerbatimEnvironment{Highlighting}{Verbatim}{commandchars=\\\{\}}
% Add ',fontsize=\small' for more characters per line
\usepackage{framed}
\definecolor{shadecolor}{RGB}{241,243,245}
\newenvironment{Shaded}{\begin{snugshade}}{\end{snugshade}}
\newcommand{\AlertTok}[1]{\textcolor[rgb]{0.68,0.00,0.00}{#1}}
\newcommand{\AnnotationTok}[1]{\textcolor[rgb]{0.37,0.37,0.37}{#1}}
\newcommand{\AttributeTok}[1]{\textcolor[rgb]{0.40,0.45,0.13}{#1}}
\newcommand{\BaseNTok}[1]{\textcolor[rgb]{0.68,0.00,0.00}{#1}}
\newcommand{\BuiltInTok}[1]{\textcolor[rgb]{0.00,0.23,0.31}{#1}}
\newcommand{\CharTok}[1]{\textcolor[rgb]{0.13,0.47,0.30}{#1}}
\newcommand{\CommentTok}[1]{\textcolor[rgb]{0.37,0.37,0.37}{#1}}
\newcommand{\CommentVarTok}[1]{\textcolor[rgb]{0.37,0.37,0.37}{\textit{#1}}}
\newcommand{\ConstantTok}[1]{\textcolor[rgb]{0.56,0.35,0.01}{#1}}
\newcommand{\ControlFlowTok}[1]{\textcolor[rgb]{0.00,0.23,0.31}{#1}}
\newcommand{\DataTypeTok}[1]{\textcolor[rgb]{0.68,0.00,0.00}{#1}}
\newcommand{\DecValTok}[1]{\textcolor[rgb]{0.68,0.00,0.00}{#1}}
\newcommand{\DocumentationTok}[1]{\textcolor[rgb]{0.37,0.37,0.37}{\textit{#1}}}
\newcommand{\ErrorTok}[1]{\textcolor[rgb]{0.68,0.00,0.00}{#1}}
\newcommand{\ExtensionTok}[1]{\textcolor[rgb]{0.00,0.23,0.31}{#1}}
\newcommand{\FloatTok}[1]{\textcolor[rgb]{0.68,0.00,0.00}{#1}}
\newcommand{\FunctionTok}[1]{\textcolor[rgb]{0.28,0.35,0.67}{#1}}
\newcommand{\ImportTok}[1]{\textcolor[rgb]{0.00,0.46,0.62}{#1}}
\newcommand{\InformationTok}[1]{\textcolor[rgb]{0.37,0.37,0.37}{#1}}
\newcommand{\KeywordTok}[1]{\textcolor[rgb]{0.00,0.23,0.31}{#1}}
\newcommand{\NormalTok}[1]{\textcolor[rgb]{0.00,0.23,0.31}{#1}}
\newcommand{\OperatorTok}[1]{\textcolor[rgb]{0.37,0.37,0.37}{#1}}
\newcommand{\OtherTok}[1]{\textcolor[rgb]{0.00,0.23,0.31}{#1}}
\newcommand{\PreprocessorTok}[1]{\textcolor[rgb]{0.68,0.00,0.00}{#1}}
\newcommand{\RegionMarkerTok}[1]{\textcolor[rgb]{0.00,0.23,0.31}{#1}}
\newcommand{\SpecialCharTok}[1]{\textcolor[rgb]{0.37,0.37,0.37}{#1}}
\newcommand{\SpecialStringTok}[1]{\textcolor[rgb]{0.13,0.47,0.30}{#1}}
\newcommand{\StringTok}[1]{\textcolor[rgb]{0.13,0.47,0.30}{#1}}
\newcommand{\VariableTok}[1]{\textcolor[rgb]{0.07,0.07,0.07}{#1}}
\newcommand{\VerbatimStringTok}[1]{\textcolor[rgb]{0.13,0.47,0.30}{#1}}
\newcommand{\WarningTok}[1]{\textcolor[rgb]{0.37,0.37,0.37}{\textit{#1}}}

\providecommand{\tightlist}{%
  \setlength{\itemsep}{0pt}\setlength{\parskip}{0pt}}\usepackage{longtable,booktabs,array}
\usepackage{calc} % for calculating minipage widths
% Correct order of tables after \paragraph or \subparagraph
\usepackage{etoolbox}
\makeatletter
\patchcmd\longtable{\par}{\if@noskipsec\mbox{}\fi\par}{}{}
\makeatother
% Allow footnotes in longtable head/foot
\IfFileExists{footnotehyper.sty}{\usepackage{footnotehyper}}{\usepackage{footnote}}
\makesavenoteenv{longtable}
\usepackage{graphicx}
\makeatletter
\def\maxwidth{\ifdim\Gin@nat@width>\linewidth\linewidth\else\Gin@nat@width\fi}
\def\maxheight{\ifdim\Gin@nat@height>\textheight\textheight\else\Gin@nat@height\fi}
\makeatother
% Scale images if necessary, so that they will not overflow the page
% margins by default, and it is still possible to overwrite the defaults
% using explicit options in \includegraphics[width, height, ...]{}
\setkeys{Gin}{width=\maxwidth,height=\maxheight,keepaspectratio}
% Set default figure placement to htbp
\makeatletter
\def\fps@figure{htbp}
\makeatother

\KOMAoption{captions}{tableheading}
\makeatletter
\makeatother
\makeatletter
\@ifpackageloaded{bookmark}{}{\usepackage{bookmark}}
\makeatother
\makeatletter
\@ifpackageloaded{caption}{}{\usepackage{caption}}
\AtBeginDocument{%
\ifdefined\contentsname
  \renewcommand*\contentsname{Table of contents}
\else
  \newcommand\contentsname{Table of contents}
\fi
\ifdefined\listfigurename
  \renewcommand*\listfigurename{List of Figures}
\else
  \newcommand\listfigurename{List of Figures}
\fi
\ifdefined\listtablename
  \renewcommand*\listtablename{List of Tables}
\else
  \newcommand\listtablename{List of Tables}
\fi
\ifdefined\figurename
  \renewcommand*\figurename{Figure}
\else
  \newcommand\figurename{Figure}
\fi
\ifdefined\tablename
  \renewcommand*\tablename{Table}
\else
  \newcommand\tablename{Table}
\fi
}
\@ifpackageloaded{float}{}{\usepackage{float}}
\floatstyle{ruled}
\@ifundefined{c@chapter}{\newfloat{codelisting}{h}{lop}}{\newfloat{codelisting}{h}{lop}[chapter]}
\floatname{codelisting}{Listing}
\newcommand*\listoflistings{\listof{codelisting}{List of Listings}}
\makeatother
\makeatletter
\@ifpackageloaded{caption}{}{\usepackage{caption}}
\@ifpackageloaded{subcaption}{}{\usepackage{subcaption}}
\makeatother
\makeatletter
\@ifpackageloaded{tcolorbox}{}{\usepackage[many]{tcolorbox}}
\makeatother
\makeatletter
\@ifundefined{shadecolor}{\definecolor{shadecolor}{rgb}{.97, .97, .97}}
\makeatother
\makeatletter
\makeatother
\ifLuaTeX
  \usepackage{selnolig}  % disable illegal ligatures
\fi
\IfFileExists{bookmark.sty}{\usepackage{bookmark}}{\usepackage{hyperref}}
\IfFileExists{xurl.sty}{\usepackage{xurl}}{} % add URL line breaks if available
\urlstyle{same} % disable monospaced font for URLs
\hypersetup{
  pdftitle={AI Assistants Maximized},
  pdfauthor={Sam Parmar},
  colorlinks=true,
  linkcolor={blue},
  filecolor={Maroon},
  citecolor={Blue},
  urlcolor={Blue},
  pdfcreator={LaTeX via pandoc}}

\title{AI Assistants Maximized}
\usepackage{etoolbox}
\makeatletter
\providecommand{\subtitle}[1]{% add subtitle to \maketitle
  \apptocmd{\@title}{\par {\large #1 \par}}{}{}
}
\makeatother
\subtitle{Tips and Tricks for R programmers}
\author{Sam Parmar}
\date{}

\begin{document}
\maketitle
\ifdefined\Shaded\renewenvironment{Shaded}{\begin{tcolorbox}[frame hidden, enhanced, boxrule=0pt, sharp corners, borderline west={3pt}{0pt}{shadecolor}, breakable, interior hidden]}{\end{tcolorbox}}\fi

\renewcommand*\contentsname{Table of contents}
{
\hypersetup{linkcolor=}
\setcounter{tocdepth}{2}
\tableofcontents
}
\bookmarksetup{startatroot}

\hypertarget{preface}{%
\chapter*{Preface}\label{preface}}
\addcontentsline{toc}{chapter}{Preface}

\markboth{Preface}{Preface}

This is a guidance book with tips and tricks for use of AI tools such as
\href{https://chat.openai.com/}{ChatGPT},
\href{https://bard.google.com/}{Bard} or
\href{https://github.com/features/copilot}{Github Copilot}. It collects
different use cases that I've run into as an R programmer. Example
prompts are also provided with a few of the use cases. This book is
created using \href{https://quarto.org/}{Quarto}. It's hosted on Github
pages, with the source code available on
\href{https:://www.github.com/parmsam/tips-and-tricks-ai-tools}{Github}.

\hypertarget{prerequisites}{%
\section*{Prerequisites}\label{prerequisites}}
\addcontentsline{toc}{section}{Prerequisites}

\markright{Prerequisites}

The book assumes you already have access to one more of the already
specified AI tools or a variant of them.

\bookmarksetup{startatroot}

\hypertarget{uses}{%
\chapter{Uses}\label{uses}}

\hypertarget{memory-aid}{%
\section{Memory aid}\label{memory-aid}}

Obtain list of possible mnemonics to help memorize a concept. These
could be an acronym or funny story. Alternatively, you could ask for a
description to help you map the information into your memory palace.

\hypertarget{example-prompt}{%
\subsection{Example prompt}\label{example-prompt}}

\begin{Shaded}
\begin{Highlighting}[]
\NormalTok{Give me a list of 5 memory devices to remember}
\NormalTok{$ vs \^{} in regular expressions}
\end{Highlighting}
\end{Shaded}

\hypertarget{transforming-information}{%
\section{Transforming information}\label{transforming-information}}

Summarize educational content into a list of bullet points or table.
Content be a video transcript or an article. Or you could go in the
opposite direction and ask for a description of a table or set of bullet
points.

\hypertarget{unit-testing}{%
\section{Unit testing}\label{unit-testing}}

Provide function(s) in your prompt and ask for unit tests. This is
useful for package development or data analyses involving custom
functions.

\begin{Shaded}
\begin{Highlighting}[]
\NormalTok{Give me few unit tests for the following R function using }
\NormalTok{the testthat package: }\InformationTok{\textasciigrave{}[insert function here]\textasciigrave{}}
\end{Highlighting}
\end{Shaded}

\hypertarget{regular-expressions}{%
\section{Regular expressions}\label{regular-expressions}}

Get a regular expression for pattern matching. Share an example string
and the subset you would like to capture then ask for a pattern to match
for it. Ensure that you create or obtain one or more unit tests to
confirm it is working as expected.

\begin{Shaded}
\begin{Highlighting}[]
\NormalTok{Regex pattern in R that looks for function names that are }
\NormalTok{not prefixed by package name. For example it should not }
\NormalTok{detect }\InformationTok{\textasciigrave{}dplyr::count(iris, Species)\textasciigrave{}}\NormalTok{ or }
\InformationTok{\textasciigrave{}dplyr::glimpse(iris)\textasciigrave{}}
\end{Highlighting}
\end{Shaded}

\hypertarget{commenting-code}{%
\section{Commenting code}\label{commenting-code}}

Add beacons or useful comments into existing code. Beacons are parts of
a script that can help a programmer understand what the code does. These
could be as a simple, one-line comment explaining a section of code or a
function. Alternatively, they could be variables using words that help
explain what the code is doing. This could mean variables like tree or
root for a code involving a binary tree.

\hypertarget{explain-concepts}{%
\section{Explain concepts}\label{explain-concepts}}

Ask AI tool to explain a new concept you are learning in a format of
interest or style of interest.

\hypertarget{write-documentation}{%
\section{Write documentation}\label{write-documentation}}

Explain what kind of R package or analysis you are working on then ask
the AI tool to create a \texttt{README.md} template for an R package.

\hypertarget{write-a-professional-email}{%
\section{Write a professional email}\label{write-a-professional-email}}

Provide an informal email or message to the AI tool and ask the tool to
make it professional based on the intended audience or recipient.

\begin{Shaded}
\begin{Highlighting}[]
\NormalTok{Please update the following email to ensure it is business }
\NormalTok{professional: }\InformationTok{\textasciigrave{}[insert email content here]\textasciigrave{}}
\end{Highlighting}
\end{Shaded}

\hypertarget{naming-things}{%
\section{Naming things}\label{naming-things}}

\begin{quote}
There are only two hard things in Computer Science: cache invalidation
and naming things. -- Phil Karlton
\end{quote}

Well, AI tools can help you name things. For example, you could ask the
AI tool to provide you could share the intended purpose for the function
you're working on then ask the AI tool for one or more informative
function names.

\bookmarksetup{startatroot}

\hypertarget{learn-more}{%
\chapter{Learn more}\label{learn-more}}

\begin{itemize}
\tightlist
\item
  \href{https://www.promptingguide.ai/}{Prompt Engineering Guide}
\item
  \href{https://writings.stephenwolfram.com/2023/02/what-is-chatgpt-doing-and-why-does-it-work/}{What
  Is ChatGPT Doing \ldots{} and Why Does It Work?}
\item
  \href{https://github.blog/2023-06-20-how-to-write-better-prompts-for-github-copilot/}{How
  to use GitHub Copilot: Prompts, tips, and use cases}
\end{itemize}



\end{document}
